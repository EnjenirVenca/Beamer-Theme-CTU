%!TEX root = ../Main.tex

\section{Overlays}

\subsection{Usage of Overlays}

\begin{frame}{Using \textbackslash pause in a List}
    The \textbackslash pause command allows you to reveal list items progressively during a presentation:
    \begin{itemize}
        \item First, show the title or context
        \pause
        \item Then introduce key concepts, one at a time
        \pause
        \item Finally, conclude with summary points or takeaways
    \end{itemize}
\end{frame}

\begin{frame}{Using \textbackslash uncover, \textbackslash visible, and \textbackslash only}
    \begin{description}
        \item[\textbackslash uncover] The content is present in the slide but hidden, occupying space.
        \item[\textbackslash visible] Similar to \textbackslash uncover, but if content is hidden, it does not show as transparent.
        \item[\textbackslash only] The content is only present on specific slides, and it does not occupy space on others.
    \end{description}
\end{frame}

\subsection{Examples of Overlay Commands}

\begin{frame}{Progressive Text Revealing with \textbackslash uncover}

    We will now reveal three pieces of information one by one:
    \begin{itemize}
        \uncover<1->{\item First piece of information revealed.}
        \uncover<2->{\item Second piece of information revealed.}
        \uncover<3->{\item Third piece of information revealed.}
    \end{itemize}
    
    As the presentation progresses, each item becomes visible at its designated step.
    
\end{frame}

\begin{frame}{Conditional Display with \textbackslash only and \textbackslash visible}

    \only<1-2>{
        This content is shown on the first and second slides only.
    }

    \visible<3-4>{
        This content is visible only on the third and fourth slides, but it occupies space on all slides.
    }

    \uncover<5>{
        This content is uncovered on the fifth slide, occupying space even when hidden on earlier slides.
    }

\end{frame}